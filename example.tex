
% example.tex — 示例
\documentclass{cpipc}
% 若系统没有方正小标宋简体,请取消下一行注释以回退到宋体:
% \PassOptionsToClass{titlecjkfont=SimSun}{cpipc}

\title{示例项目}
\version{V1.0}
\teamname{某某大学AI实验室}
\category{创新应用组}

\begin{document}

% 封面(前三页采用罗马数字编号)
\maketitle

% 目录页
\makemycontents

% 记录更改历史页
\changehistory

% 从此处开始切换为阿拉伯数字页码
\startmain

% 正文
\section{项目概况}

\subsection{背景和基础}

简单阐述项目起因和已有工作基础,包括项目灵感、团队构成等。英文/Latin 内容将自动使用 Times New Roman。

本项目基于深度学习技术\cite{goodfellow2016deep},旨在解决实际问题。团队成员具有丰富的AI研发经验。深度学习在现代AI应用中发挥着重要作用\cite{lecun2015deep}。

\subsection{场景和价值}

简单阐述该项目适用的应用场景及潜在社会价值,包括市场调研、对比性分析等。

该项目具有广泛的应用前景,能够为相关行业带来显著的效益提升。

\subsection{所需支持}

请阐述项目实施过程中所需支持,如算力、硬件、相关培训等。

项目需要高性能计算资源和专业的技术指导。

\section{项目规划}
\subsection{整体目标}

简单阐述参赛期间本项目的整体目标,比如可展示原型系统、在行业中初步验证等。

我们的目标是在比赛期间完成一个可演示的原型系统。

\subsection{技术创新点}

通过相关技术对比调研,简单阐述项目的主要技术创新点。

本项目的创新点主要体现在算法优化和应用场景的结合上。我们采用了先进的残差网络架构\cite{he2016deep},并结合了最新的研究成果\cite{zhang2023deep,li2023neural}。

\section{实施方案}
\subsection{技术可行性分析}

请阐述数据采集、行业知识获取,以及算力、硬件支持等可行性分析。

经过前期调研,我们已经确认了数据来源和技术路线的可行性。

\subsection{技术细节}

请阐述相关技术细节,并结合可行性分析对核心技术进行论证及预期技术指标。

核心技术采用最新的深度学习架构,预期能够达到业界先进水平。

\subsection{计划和分工}

结合参赛时间点,简单阐述本项目的整体计划和团队分工。

项目分为三个阶段实施,每个阶段都有明确的里程碑和负责人。



\section{参考资料}

本项目参考了以下重要文献和资源。LaTeX的使用可以参考《LaTeX入门》\cite{liuhaiyang2013latex}。

% 使用BibTeX生成参考文献,不显示默认标题
\renewcommand{\refname}{}% 移除BibTeX默认的"参考文献"标题
\bibliographystyle{plain}
\bibliography{example}

\end{document}
