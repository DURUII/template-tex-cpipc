
% example.tex — 示例
\documentclass{cpipc}
% 若系统没有方正小标宋简体,请取消下一行注释以回退到宋体:
% \PassOptionsToClass{titlecjkfont=SimSun}{ai-innovation}

\setprojectname{示例项目}
\setversion{V1.0}
\setprojectdate{2025.08.27}
\setteam{某某大学AI实验室}
\setcategory{创新应用组}

\begin{document}

% 封面(前三页采用罗马数字编号)
\maketitlepage

% 目录页
\makemycontents

% 记录更改历史页
\changehistory

% 从此处开始切换为阿拉伯数字页码
\startmain

% 正文
{\SectionHeading\section{项目概况}}

{\SubsectionHeading\subsection{背景和基础}}
{\BodyText
简单阐述项目起因和已有工作基础,包括项目灵感、团队构成等。英文/Latin 内容将自动使用 Times New Roman。
}

{\SubsectionHeading\subsection{场景和价值}}
{\BodyText
简单阐述该项目适用的应用场景及潜在社会价值,包括市场调研、对比性分析等。
}

{\SubsectionHeading\subsection{所需支持}}
{\BodyText
请阐述项目实施过程中所需支持,如算力、硬件、相关培训等。
}

{\SectionHeading\section{项目规划}}
{\SubsectionHeading\subsection{整体目标}}
{\BodyText
简单阐述参赛期间本项目的整体目标,比如可展示原型系统、在行业中初步验证等。
}

{\SubsectionHeading\subsection{技术创新点}}
{\BodyText
通过相关技术对比调研,简单阐述项目的主要技术创新点。
}

{\SectionHeading\section{实施方案}}
{\SubsectionHeading\subsection{技术可行性分析}}
{\BodyText
请阐述数据采集、行业知识获取,以及算力、硬件支持等可行性分析。
}

{\SubsectionHeading\subsection{技术细节}}
{\BodyText
请阐述相关技术细节,并结合可行性分析对核心技术进行论证及预期技术指标。
}

{\SubsectionHeading\subsection{计划和分工}}
{\BodyText
结合参赛时间点,简单阐述本项目的整体计划和团队分工。
}

{\SectionHeading\section{参考资料}}
{\BodyText
按规范列出参考文献条目。
}

\end{document}
