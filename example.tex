
% example.tex — 示例
\documentclass{cpipc}
% 若系统没有方正小标宋简体,请取消下一行注释以回退到宋体:
% \PassOptionsToClass{titlecjkfont=SimSun}{ai-innovation}

\setprojectname{示例项目}
\setversion{V1.0}
\setprojectdate{2025.08.27}
\setteam{某某大学AI实验室}
\setcategory{创新应用组}

\begin{document}

% 封面(前三页采用罗马数字编号)
\maketitlepage

% 目录页
\makemycontents

% 记录更改历史页
\changehistory

% 从此处开始切换为阿拉伯数字页码
\startmain

% 正文
\section{项目概况}

\subsection{背景和基础}
{\BodyText
\FirstParagraph 简单阐述项目起因和已有工作基础,包括项目灵感、团队构成等。英文/Latin 内容将自动使用 Times New Roman。

\NextParagraph 本项目基于深度学习技术,旨在解决实际问题。团队成员具有丰富的AI研发经验。
}

\subsection{场景和价值}
{\BodyText
\FirstParagraph 简单阐述该项目适用的应用场景及潜在社会价值,包括市场调研、对比性分析等。

\NextParagraph 该项目具有广泛的应用前景,能够为相关行业带来显著的效益提升。
}

\subsection{所需支持}
{\BodyText
\FirstParagraph 请阐述项目实施过程中所需支持,如算力、硬件、相关培训等。

\NextParagraph 项目需要高性能计算资源和专业的技术指导。
}

\section{项目规划}
\subsection{整体目标}
{\BodyText
\FirstParagraph 简单阐述参赛期间本项目的整体目标,比如可展示原型系统、在行业中初步验证等。

\NextParagraph 我们的目标是在比赛期间完成一个可演示的原型系统。
}

\subsection{技术创新点}
{\BodyText
\FirstParagraph 通过相关技术对比调研,简单阐述项目的主要技术创新点。

\NextParagraph 本项目的创新点主要体现在算法优化和应用场景的结合上。
}

\section{实施方案}
\subsection{技术可行性分析}
{\BodyText
\FirstParagraph 请阐述数据采集、行业知识获取,以及算力、硬件支持等可行性分析。

\NextParagraph 经过前期调研,我们已经确认了数据来源和技术路线的可行性。
}

\subsection{技术细节}
{\BodyText
\FirstParagraph 请阐述相关技术细节,并结合可行性分析对核心技术进行论证及预期技术指标。

\NextParagraph 核心技术采用最新的深度学习架构,预期能够达到业界先进水平。
}

\subsection{计划和分工}
{\BodyText
\FirstParagraph 结合参赛时间点,简单阐述本项目的整体计划和团队分工。

\NextParagraph 项目分为三个阶段实施,每个阶段都有明确的里程碑和负责人。
}

\section{参考资料}
{\BodyText
\FirstParagraph 按规范列出参考文献条目。

\NextParagraph [1] Zhang, S., et al. "Deep Learning for AI Innovation." ICML 2023.

\NextParagraph [2] Li, M., et al. "Advanced Neural Networks." Nature AI 2023.
}

\end{document}
